%!TEX root = ../dissertation.tex
\chapter{Annex}
\label{appendixA}

\section{Choosing the right tool for the job: a tutorial (more of an annex, maybe?)}
% \addcontentsline{toc}{section}{Choosing the right tool for the job: a tutorial (more of an annex, maybe?)}
While we can go deeper and deeper in analysing more elementary particles than protons, neutrons and electrons, we can stop there because the models we can build with them are good enough for the kind of chemistry we want to study. As a matter of fact, we can just consider electrons and nuclei, if not just whole atoms!

\textbf{First question: What do you want to study?}

XXXXXX

\textbf{Second question: What theory model will you (or can you) apply?}

Accuracy is intrinsically linked to computational power and both condition the dimensionality of the system one could study. This can be easily depicted by describing the physicochemical models used to study atoms in the modelling of biochemical systems, which are basically divided in two families: quantum mechanics (QM) and molecular mechanics (MM). A way to summarise their differences could be that QM methods deal explicitly with both electrons and nuclei solving the Schrödinger equations, while MM methods stand on a description of atoms as charged single-point masses with simplified (but not necessarily simple) sets of parameters for both covalent and non-covalent interactions between them following harmonic oscillator approximations – the force field. As opposed to QM models, the classical representation of atoms and bonds in standard MM methods do not allow to deal with chemical reactivity, but those simplifications allow to solve the energy of a given system using several orders of magnitude fewer resources than the QM ones.

\section{Programming languages history: from punch cards to assembly to C to Python. }
% \addcontentsline{toc}{section}{Programming languages history: from punch cards to assembly to C to Python. }
History in some of the links here: \href{https://www.levenez.com/lang/}{https://www.levenez.com/lang/}. Also wiki: \href{https://en.wikipedia.org/wiki/History\_of\_programming\_languages}{https://en.wikipedia.org/wiki/History\_of\_programming\_languages}.

Figure! One of these:

\begin{itemize}
	\item \href{http://archive.oreilly.com/pub/a/oreilly/news/languageposter\_0504.html}{http://archive.oreilly.com/pub/a/oreilly/news/languageposter\_0504.html}

	\item \href{https://github.com/stereobooster/programming-languages-genealogical-tree}{https://github.com/stereobooster/programming-languages-genealogical-tree}
\end{itemize}


