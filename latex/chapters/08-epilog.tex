%!TEX root = ../dissertation.tex
\chapter*{Epilog}
\addcontentsline{toc}{chapter}{Epilog}
\label{chap:epilog}

During the development of new software, difficulties can arise anytime, for any reason. Dependencies, installation and distribution are inherent problems to the complex landscape of libraries, operating systems and hardware architectures. Solving them efficiently requires using developer-specific tools, usually disregarded by end-users. I began my Ph. D. studies as a user and ended up as some sort of developer, and to my surprise, my most popular project is not GaudiMM itself, but a tool created as a helper for its development: PyChimera. The need for interconnected software is patent, and a big part of this dissertation has been devoted to bringing new free alternatives to the table. The Tangram suite is only an attempt at providing molecular modeling tools accessible for beginner users that do not want to mess up with complex installations and input files: the workflow has been designed to be intuitive and consistent.

Still, most complex tasks would require some sort of scripting for an efficient solution. Programming skills are essential in all fields of science that can be enhanced by computational support. Two main properties can be identified: they can accelerate repetitive tasks, freeing time for other problems, and at the same time they enable new problem-solving strategies and can help plan studies in a different way. In other words, programming skills streamline creative thinking.

GaudiMM was a simple (but inefficient) for-loop doing conditional operations. It was the abstraction that followed that unlocked new possible workflows. New questions could be asked, new ideas could be devised. \textit{Is it possible to...?} started to have new endings.

Yet we are failing to convince students why coding is important. Most still prefer to go manually to \textit{make sure} results will be correct, and not a false positive product of a non-obvious bug. Is this due to the uninformative error messages, or does it go deeper in our educational system roots? Students are still taught memorization and mechanization, but that should not be the point in this era. Computers are much better at that than us, and will surpass us in other areas too. Solving problems is not copying algorithms and following instructions. It should be more about the reasons behind each of those steps. Designing algorithms, protocols, tools and frameworks: that should be the goal. Otherwise, the inability to write code will become the illiteracy of the XXI\textsuperscript{st} century.