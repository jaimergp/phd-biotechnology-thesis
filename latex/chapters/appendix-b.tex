\chapter{Undergoing developments in GaudiMM}
\label{chap:appendix-b}
% \addcontentsline{toc}{section}{GaudiMM as an educational tool: undergoing developments}


\section{Navigating the chemical space}
% \addcontentsline{toc}{section}{Navigating the chemical space}
GaudiMM already allowed to navigate the chemical space via the dynamic building capabilities of the Molecule gene, but it presented two limitations: (1) it is restricted to the provided fragments library, and (2) it only allows to construct linear concatenations of those fragments (i.e. no ramifications or rings).

A new approach based on graph theory and pharmacophore matching is being developed in our group as part of the Ph.D. thesis of J. E. Sánchez-Aparicio. This method, which interprets molecules as non-directed graphs that can grow and shrink arbitrarily, does not require any preexisting libraries and naturally considers ramifications. It has been successfully applied to propose designs of small molecule inhibitors for \textit{K. pulmoniae} NDM-1 $ \beta $ -lactamase.

\section{Finding ligand binding pathways}
% \addcontentsline{toc}{section}{Finding ligand binding pathways}
Docking studies provides insight on how a small molecule can interact with a bigger host molecule by assessing feasible binding poses. However, those are just static snapshots of a dynamic behavior. To study how the ligand reaches its binding sites, long molecular dynamics runs with steering restraints are needed and do not always guarantee a successful ligand pathway.

An alternative approach was considered for one of the MSc dissertations supervised during this Ph.D. The protein space was flooded with small probes placed in a tight grid and queried for steric impediments, resulting in points with higher or lower pseudo-energy scores. Then, lower-energy points were traversed from the outer regions of the protein in hopes of finding a continuous path that reached the ligand binding site. To consider the ligand size, shape or volume, a second step was proposed. The calculated paths were segmented in 5Å pieces and each of the resulting pieces was then submitted to a docking simulation with reduced search radius. The resulting structures were low-energy conformations of the ligand along the proposed pathway. All these poses were finally concatenated together to emulate a smooth trajectory ideal for depiction purposes.

This proof of concept proves how the versatility present in GaudiMM can be used as part of bigger protocols, and is being reimplemented as a gene able to guide the exploration of docking studies along feasible pathways in the Ph.D. studies of J. E. Sánchez-Aparicio.
