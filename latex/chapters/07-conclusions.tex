%!TEX root = ../dissertation.tex
\chapter{General conclusions}
\label{chap:07}

\Lettrine{In this dissertation} a series of pieces of software has been presented and several applications have been brought. The computational platforms have been developed thanks to the unique modularity that offers Python. A possible list of conclusive arguments could be the following:

% This approach useful for both molecular modelers and experimentalists.


\begin{enumerate}
	\item GaudiMM has been presented as a versatile molecular optimization framework[DETAILS ON ARCHITETURE AND NOVEL PARADIGM]. Its uncoupled plurigenetic, multiobjective implementation provides researchers an unprecedented flexibility in molecular modeling.  The concept of recipe ease a possible path to perform hypothesis-driven modeling as well as other simulations like docking or restrained exploration.

	\item Tangram is a collection of more than 15 tools for UCSF Chimera that will help in the generation of input files for 3\textsuperscript{rd} party software and diverse interactive structural analysis within a single graphical interface and user experience.

	\item OMMProtocol provides a user-friendly, single-file interface to the powerful, GPU-accelerated OpenMM molecular dynamics libraries, which has brought a 20-fold speed increase to the previously available MD protocols.

	\item Garleek will help in those QM/MM studies that require extended molecular mechanics force fields. By seamlessly interfacing Gaussian with modern MM suites, more accurate calculations can be obtained.

	\item ESIgen can save hours of manual text manipulation in computational chemistry. Its ability to automatically generate technical reports suitable for attachment as supporting information or internal communication with colleagues will be hopefully appreciated by this community.

	\item EasyMECP is a very specific tool designed to facilitate the calculation of minimum energy crossing points (MECP) with Gaussian. It uses the popular code by J. M. Harvey, but provides a user-friendly wrapper that alleviates the job set-up significantly.\todo{Maybe adieu}
\end{enumerate}

% Unfortunately, some unfinished projects could not make it to this compilation but will constitute further steps in the future of these ongoing endeavours. At the end of the day, even if only one of these tools end up in the workflow of a future molecular modeler, the effort will have been worth it.

Finally, these last following paragraphs will try to sum up some points learned that are not directly linked to the results presented, but to the process of obtaining them.

Initially, the ‘novel software platform’ in the title of this dissertation only referred to GaudiMM, but during its development several weak points in the workflow of lab-mates and colleagues were identified. What started as innocent efforts to help automate some manual steps or provide easier access to new technology ended up as big projects on their own. This kind of opportunities was not anticipated, but it perfectly fitted the intentions of this thesis.

- IDEAS

	- He podido dar los primeros pasos a un tipo de desarrollo capaz de competir en funcionalidad con suites comerciales ya existentes (particularmente caras en algunos casos).

	- Además de reproducir resultados ya existentes, este tipo de plataformas ha permitido entrar en temas de modeling donde no hay inputs (huérfanos). ES un approach flexible capaz de mezclar descriptores sencillos con energías de coordinación para proporcionar soluciones no caracterizables previamente (al menos sin una inversión importante de recursos computacionales, o tiempo).

	- LISTA: casos más interesantes comentados en chapter 06.

	The conceptual separation of exploration and evaluation implemented in GaudiMM gives a clear understanding of the different variables involved in the optimization process. This has proved to be a very valuable as a teaching tool in lower degrees of education. Students involved in GaudiMM development have contributed new modules even with a non-chemical background. Some highlights include a gene to navigate the chemical space or a coupled gene/objective pair to assess ligand binding pathways, detailed in Appendix X.


