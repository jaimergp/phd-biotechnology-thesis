%!TEX root = ../dissertation.tex
\chapter{General conclusions}
\label{chap:07}

\Lettrine{In this dissertation} several computational tools have been presented and several applications have been benchmarked and showcased. Globally, the list of achievements could be summarized in six points:

% This approach useful for both molecular modelers and experimentalists.


\begin{enumerate}
	\item \textsc{GaudiMM} has been presented as a versatile molecular optimization framework with high modularity. Its uncoupled plurigenetic, multi-objective implementation provides researchers an unprecedented flexibility in molecular modeling. Instead of conforming to the requirements of a sequential multistep protocol, the same methods can work synergistically in the same modeling exercise. The concept of \textit{recipe} paves the way towards performing hypothesis-driven modeling as well as other simulations like dockings or restrained conformational exploration.

	\item \textsc{Tangram} is a collection of more than 15 tools for UCSF Chimera that will help in the generation of input files for 3\textsuperscript{rd} party software and diverse interactive structural analysis within a single graphical interface and user experience.

	\item \textsc{OMMProtocol} provides a user-friendly, single-file interface to the powerful, GPU-accelerated OpenMM molecular dynamics libraries. These tools have brought a 20-fold speed increase to the previously followed MD protocols in our group.

	\item \textsc{Garleek} has been designed to help in those QM/MM studies that require extended molecular mechanics force fields. By seamlessly interfacing Gaussian with modern MM suites, more accurate calculations can be obtained.

	\item \textsc{ESIgen} can save hours of manual text manipulation in computational chemistry. Its ability to automatically generate technical reports suitable for attachment as supporting information documents or internal communication with colleagues will be hopefully appreciated by this community. Computational chemists will also welcome \textsc{EasyMECP}, designed to facilitate the calculation of minimum energy crossing points (MECP) with Gaussian.

\end{enumerate}


These ongoing efforts have been the first steps towards developing a suite able to compete, feature-wise, with available commercial suites ---which can be particularly expensive in some cases--- at no cost for academics.

\subsection*{On GaudiMM}

In addition to reproducing and benchmarking known problems, this platform has been able to model orphan systems where currently available information is scarce. This is thanks to a versatile approach: creating optimization synergies between deliberate simplistic chemical and geometric descriptors. Some tasks that have benefitted from this idea are:

\begin{itemize}
	\item \textsc{Exotic docking prediction}. GaudiMM expands the possibilities of docking calculations beyond the traditional flexible protein-ligand dockings, enabling unconventional docking studies like competitive docking or multicovalent restraints (see \autoref[section]{section:proteinliganddocking} for a benchmark on standard protein-ligand docking and the take on more exotic cases).
	\item \textsc{Complex molecular design}. Predicting possible structures of partially characterized systems by performing hypothesis-driven modeling (see \autoref[section]{section:rotaxane} on multivalent enzyme inhibition). This includes designing complex ligands where only some experimental information is available, if any (see \autoref[section]{section:dibiotin-linker-length-optimization} for the optimization of a dibiotin ligand).
	\item \textsc{Finding metal binding sites in proteins}. Modeling organic systems where metal-residue interactions can be expressed with coordination geometries (see \autoref[section]{section:metal-applications} for this and other cases of coordination-driven folding).
\end{itemize}


Additionally, the conceptual separation of exploration and evaluation as implemented in GaudiMM gives a clear understanding of the different variables involved in an optimization process. This has proved to be a very valuable as a teaching tool in lower degrees of education. Students involved in GaudiMM development have contributed new modules even with a non-chemical background. Some highlights include a gene to navigate the chemical space or a coupled gene/objective pair to assess ligand binding pathways, detailed in \autoref[appendix]{chap:appendix-b}.

Of course, there is further work to do. GaudiMM's approach has a modestly steep learning curve and configuring an input file is mostly done on the text editor. A general-purpose graphic interface would be desirable and is something to consider. In the short term, the concept of application-specific interfaces is very attractive (e.g. searching metal binding sites or optimizing the length of linkers).

Analyzing results from a multi-objective process can be daunting at first because considering the optimality of two or more criteria simultaneously is not intuitive. While GaudiView is available to perform sorting and filtering on the candidate solutions, certain applications could benefit from a unified scoring term. However, this would require constructing a weighted linear sum of the objectives by benchmarking big datasets. In that matter, machine learning approaches could be very helpful.


\section*{On Python}

Without Python and its great ecosystem (UCSF Chimera, the SciPy stack and the Omnia project have been particularly important) this dissertation would not have been possible. All the developments carried out during this Ph.D. are the consequences of its unique vision.

The \textit{de facto} Python installation already provides a library for high-level operations, freeing the developer from dealing with technical nuances. Beyond the official distribution, the catalog of ready-to-use packages is excellent, allowing to prototype projects in very little time just by importing the needed requirements. This is particularly true in scientific software, where it shines as the perfect glue language to stick different projects together.

Moreover, the emphasis on readability and self-documented code contributes to maintaining good practices along the full development cycle, even when different people are involved. This is particularly important for long-lasting efforts in research and fruitful investment in research.

This Ph.D. hopefully illustrates how Python and its exceptional ecosystem offer molecular modelers with a versatile canvas for innovative science.
