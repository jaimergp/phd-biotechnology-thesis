\subsection{Living with metal ions in macromolecules}
% \addcontentsline{toc}{subsection}{Living with metal ions in macromolecules}
One of the most exciting areas of molecular modeling sits at the frontier between organometallics and biochemistry, two fields that have been studied separately in computational chemistry for decades now. Globally, chemists exploit their features differently and, as such, present different computational challenges. Traditionally, organometallic systems feature a reduced number of atoms and accommodate transition metal centers within their structure, whose exotic electronic behavior can only be accurately computed with quantum chemistry approaches. Studies on biological problems such as the early work on folding of peptides and proteins had to face a larger number of atoms (hundreds or thousands) from the beginning, forcing the authors to use classical mechanics approaches to deal with the added dimensionality after realizing that the electronics of the system were not very important in that process.

However, metals do take part in biological processes as mainstream as oxygen transportation and muscle contraction. As such, the existence of metalloproteins cannot be neglected by the modeling community, who should bring these two areas together in a more seamless experience. Given the diverging efforts accumulated for decades, the gap is not easily overcome, but some solutions exist. Depending on the properties to study, one can resort to different approaches, as detailed below.

\subsubsection{Quantum Mechanics}
% \addcontentsline{toc}{subsubsection}{Quantum Mechanics}
Since quantum mechanics deal explicitly with the electronic shells of atoms, the immense diversity of electronic configurations of metal ions does not represent a problem. If such, the only challenge this might present is choosing the adequate functionals, basis sets or starting-point structures.

The challenge is more technical than scientific. While advances in DFT theories and hardware architectures allow us to deal with up to 500-atom systems in feasible timescales, this is still far from the number of atoms usually present in protein structures. For this, hybrid QM/MM studies are more adequate: the QM layer is responsible for dealing with the metal and its surroundings (at least, the first coordination sphere), while the, comparatively cheap, MM layer governs the rest of the structure. Even with this approach, time-dependent schemes still represent a huge computational effort, not to mention the difficulties in setting up the system adequately. One must still deal with layer boundaries effects or the parameterization needed for the MM calculations.

\subsubsection{Molecular Mechanics}
% \addcontentsline{toc}{subsubsection}{Molecular Mechanics}
Sometimes, QM is not necessary for a modeling study, since the metal might only play a structural role without exhibiting reactivity. In these cases, it is more interesting to gather an insight into the structural behavior of the system along time. Nowadays, for macromolecular systems, this is only feasible with molecular dynamics approaches, which require accurately parameterized force fields. Traditionally, force fields were developed to solve problems existing with proteins, nucleic acids and organic compounds,\cite{lifson1968consistent,allinger1973,momany1975energy} so historically transition metals have not been considered in force field development. Additionally, they present complexities not present in the reduced set of organic elements: several coordination geometries, different charge states, exotic polarizable behavior... As a result, dealing with metals in molecular mechanics is usually challenging. One must choose between (1) not considering them at all, (2) using a low-accuracy general-purpose force field, or (3) facing the tedious process of parameterization.

Ignoring or removing the metal ions can be acceptable in certain cases where they do not play a crucial role in the structure or dynamics of the system, but that is rarely the case. While general purpose force fields are numerous and heavily used, they mostly target organic compounds (such as CGenFF,\cite{Vanommeslaeghe2009} GAFF,\cite{Wang2004} Tripos 5.2 force field\cite{clark1989}). Only some include parameters for metal ions: UFF (for Universal Force Field,\cite{Rappe1992} MMFF\cite{halgren1996}) covers the full periodic table, but Dreiding\cite{Mayo1990} only contains parameters for Na, Ca, Zn and Fe. While useful for organic chemistry, they are not as used in simulations including biological systems,\todo{Citation needed} since they tend to rely on the Lennard-Jones based \textit{nonbonded model}.\todo{Citation needed}

A feasible alternative for bio-containing systems is the so-called \textit{bonded model}, which treats metal ion interactions with both bonded and non-bonded parameters; i. e., the metal is assumed to bond to some residues. Some of the protein-oriented force fields like AMBER\cite{amber} or CHARMM\cite{brooks1983} distribute force field extensions for some of the most common metal ions in proteins, such as hemo-coordinated iron, but mainly as examples on how custom parameters can be added in the software. These types of force field extensions are only valid for the context where the parameters were obtained; i. e., the iron parameters for the heme groups will not reproduce the behavior of iron in other organic contexts such as ferrocenes. While the file format is easily understood, the values of the parameters are not easy to obtain: one has to resort to experimental data or \textit{ab initio} calculations to get adequate constants for bonded (distances, angles, dihedrals) and nonbonded (electrostatic, Van der Waals) interactions. While an expert user can decide to obtain those values manually, the process is not trivial and some protocols and tools have appeared to assis. They are mostly based on the Seminario’s method and his FUERZA software,\cite{Seminario1996} such as MCPB, MCPB.py,\cite{li2016} VFDFT.\cite{zheng2016} Recently, alternative approaches based on machine and statistical learning, \cite{fracchia2017,li2017b} and non-Seminario strategies\cite{Burger2012, allen2017} have also appeared, but the principle remains the same: extract the information from \textit{ab initio} calculations. Given the complexity of the task, some specific force fields have arisen lately to provide parameters for certain metals:

\begin{itemize}
	\item \href{http://onlinelibrary.wiley.com/doi/10.1002/9780470125830.ch2/summary}{http://onlinelibrary.wiley.com/doi/10.1002/9780470125830.ch2/summary}

	\item \href{http://onlinelibrary.wiley.com/doi/10.1002/jcc.10171/full}{http://onlinelibrary.wiley.com/doi/10.1002/jcc.10171/full}

	\item \href{http://pubs.acs.org/doi/abs/10.1021/ic00068a012?journalCode=inocaj}{http://pubs.acs.org/doi/abs/10.1021/ic00068a012?journalCode=inocaj}

	\item \href{http://pubs.acs.org/doi/abs/10.1021/jp046244d}{http://pubs.acs.org/doi/abs/10.1021/jp046244d}

	\item \href{http://pubs.acs.org/doi/abs/10.1021/ja00001a001?journalCode=jacsat}{http://pubs.acs.org/doi/abs/10.1021/ja00001a001?journalCode=jacsat}

	\item \href{http://onlinelibrary.wiley.com/doi/10.1002/jcc.20634/full}{http://onlinelibrary.wiley.com/doi/10.1002/jcc.20634/full}

	\item \href{http://onlinelibrary.wiley.com/doi/10.1002/pssb.201248460/full}{http://onlinelibrary.wiley.com/doi/10.1002/pssb.201248460/full}

	\item \href{http://pubs.acs.org/doi/abs/10.1021/ct400952t}{http://pubs.acs.org/doi/abs/10.1021/ct400952t}
\end{itemize}

A radically different strategy consists of mimicking the interactions of the metal site with positively-charged pseudoatoms strategically placed at around 0.9 \AA from the metal nucleus following the vertices of the adequate coordination geometry. The Cationic Dummy Atom Model (CDAM) was introduced for Mn2+ ions by Aaqvist $\&$  Warshel in 1990\cite{aaqvist1990} and has been successfully implemented in further studies fore Zn, Mg, Ca, Fe, Co, Ni, Cu and more.\cite{duarte2014,lu2012proteins,Oelschlaeger_2007,Saxena_2013,Saxena_2014,Liao_2015,Pang_1999} Among its advantages, once parameterized the CDAM approach is context-independent, but it forces a fixed coordination number and geometry on the modeled metal site.

The application of polarizable force fields (Fluctuating Charge methods, ABEEM, Drude oscillators and rods, induced dipoles, AMOEBA, PFF) or more exotic models based on Angular Overlap and Valence Bond Theory are also promising approaches, but the additional calculations incur in a big performance penalty when compared to other strategies and still require additional parameterization. Further details on the topic can be found in the extensive review published by Li and Merz Jr. in 2017.\cite{li2017}

\subsubsection{Lower levels of theory}
% \addcontentsline{toc}{subsubsection}{Lower levels of theory}
If the study at hand does not require a molecular mechanics treatment, such as docking studies of virtual screening approaches, the parameterization problem is usually not present or, at least, not that complicated. Docking studies, which try to accommodate small compounds within macromolecules, have not considered metals for years, since they were originally designed to find drug-like, organic compounds suitable for the pharmaceutical industry. Fortunately, over time some of the most popular docking packages have included strategies to deal with metals,\cite{flexx} albeit sometimes they could only be part of the host (usually a protein), and not part of the probe (the ligand).\cite{verdonk2003improved} To overcome the problem, approaches inspired in the Cationic Dummy Atom Model implemented in MM studies have been designed (\textit{H-bond trick}): in this case, the dummy atoms are hydrogen atoms that behave as a hydrogen bond donor, a chemical feature commonly implemented in docking software.

Other approaches involve considering the metal problem as a geometric optimization problem, restraining their position with distances, angles and dihedrals measurements. This strategy is partially implemented in homology modeling software like MODELLER, \cite{Sali1993} and is one of the main features of the developments presented in this thesis (detailed in \autoref{chap:03}).

In cheminformatics, explicit consideration of atoms is not as important and strategies like the pharmacophoric studies only have to consider metals as a custom type of interaction hotspot.\cite{johns2009,kawasuji2012,carcelli2014,yang2016} In QSAR, a catalogue of metal empirical properties is enough to build the dataset.\cite{walker2012fundamental}

