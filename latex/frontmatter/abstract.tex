%!TEX root = ../dissertation.tex
% the abstract


In this dissertation, a series of novel computational modeling tools is reported. All of them have been written in Python and include: (1) GaudiMM, (2) Tangram, and (3) a collection of command-line applications. This Ph.D. demonstrates the power of this unique high-level language, particularly in software development for molecular modeling.

\begin{enumerate}
    \item GaudiMM allows to build and refine chemobiological structures through a multi-objective genetic algorithm. It features a modular, extensible architecture that can be applied to diverse molecular modeling exercises, depending on the modules chosen.

    \item Tangram is a collection of graphical interfaces for UCSF Chimera. Some of these extensions provide interactive methods for setting up calculations in external programs, like Quantum Mechanics in Gaussian or Molecular Dynamics in OpenMM. Others rely on the interactive 3D viewer to depict properties of molecular structures as calculated previously in other software, turning UCSF Chimera into an even more versatile analysis tool.

    \item A variety of command-line tools has been also developed along GaudiMM and Tangram. They are mainly concerned with optimizing common workflows in molecular modeling, like running GPU-accelerated Molecular Dynamics simulations (OMMProtocol), extending the force fields used in QM/MM approaches (Garleek), or automating the elaboration of Supporting Information documents for computational chemistry calculations (ESIgen).
\end{enumerate}

To prove their usage and applicability in molecular modeling, a series of illustrative cases will be described in detail. These include toy examples that showcase the potentiality of GaudiMM ---some of them unreachable with standard methodologies---, like siderophore chelation, standard and exotic docking protocols, ligand design and metal binding site prediction.