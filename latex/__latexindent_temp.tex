@article{doi:10.1002/wcms.1320,
author = {De Vivo Marco and Cavalli Andrea},
title = {Recent advances in dynamic docking for drug discovery},
journal = {Wiley Interdisciplinary Reviews: Computational Molecular Science},
volume = {7},
number = {6},
pages = {e1320},
doi = {10.1002/wcms.1320},
url = {https://onlinelibrary.wiley.com/doi/abs/10.1002/wcms.1320},
eprint = {https://onlinelibrary.wiley.com/doi/pdf/10.1002/wcms.1320},
abstract = {Molecular docking allows the evaluation of ligand-target complementarity. This is the crucial first step in small-molecule drug discovery. Over the last decade, increasing computer power and more efficient molecular dynamics (MD) software have prompted the use of MD for molecular docking. The resulting dynamic docking offers major improvements by (1) taking full account of the structural flexibility of the drug-target system and (2) allowing the computation of the free energy and kinetics associated with drug binding. Here, we examine the recent advances in dynamic docking, while also considering the challenges and limitations that this powerful approach must overcome to impact fast-paced drug discovery. WIREs Comput Mol Sci 2017, 7:e1320. doi: 10.1002/wcms.1320 This article is categorized under: Structure and Mechanism > Computational Biochemistry and Biophysics Molecular and Statistical Mechanics > Molecular Mechanics Software > Molecular Modeling}
}